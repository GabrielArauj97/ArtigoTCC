%\thispagestyle{empty}
\section{MATERIAIS E MÉTODOS}
\subsection{HTML}
HTML é a sigla para “HyperText Markup Language”, que significa “Linguagem de Marcação de Hipertexto”. Criada por Tim Barners Lee na década de 1990, a linguagem de marcação é atualmente controlada e mantida pela W3C (World Wide Web Consortium). É uma linguagem de marcação usada para desenvolvimento de paginas web que permite a criação de documentos web que podem ser lidos por praticamente qualquer computador e são facilmente transmitidos pela Internet.
Os códigos ou tags, como são conhecidos, servem para indicar o que são cada elemento presente na página web e quais suas funções. Essas tags indicam como o texto deve ser formatado, seja por meio de parágrafos, tabelas, títulos, imagens entre outros. Os navegadores identificam essas tags e apresentam a página de acordo como está especificada. Um documento do tipo HTML é um texto simples e que pode ser editado em qualquer editor de texto \cite{Andrei2019}.

\subsection{JAVASCRIPT}
JavaScript é uma linguagem de programação criada por Brandan Eich em 1995. Primeiramente foi chamada de Mocha, e também recebeu nomes como Mona e LiveScript antes de ser chamada de JavaScript, que é como é conhecida atualmente. Inicialmente, JavaScript era limitado e de uso exclusivo da Netscape, empresa onde Eich atuava como especialista em sistemas para computadores \cite{Andrei2019a}. 
Atualmente o JavaScript é uma das linguagens de programação mais usadas no mundo, estando presente em vários navegadores e sistemas operacionais de dispositivos moveis e desktops. Dados de 2016 apontam que a linguagem é usada por mais de 92% dos sites presentes na Internet, o que mostra que em pouco mais de 20 anos após ter sido criada, é uma das ferramentas mais importantes da programação web.
Esse alto uso da linguagem pode ser explicado pela sua variada usabilidade e facilidade, pois o JavaScript, ou JS como é popularmente conhecido, é capaz de criar animações, mapas interativos, gráficos animados em 2 ou 3 dimensões e aplicativos para dispositivos moveis, além de muitas outras aplicações. A linguagem é capaz de controlar os elementos presentes na página, por exemplo, imagine um sistema onde existe um relógio analógico onde os ponteiros se movem de acordo com que o tempo passa. O movimentos dos ponteiros desse relógio pode ser feito usando JavaScript.

\subsection{PHP}
O PHP é uma linguagem de programação web criada pelo programador dinamarquês Rasmus Lerdorf, que usava um conjunto de códigos binários escritos em linguagem C, para fazer conexão entre sistemas e servidores através da Internet. Inicialmente, esse conjunto de scripts, era usado por Rasmus para verificar a quantidade de acessos ao seu currículo, presente em seu site pessoal. 
Por ser uma linguagem composta por scripts, o PHP trabalha em conjunto com o HTML. A conexão entre a linguagem de programação e a linguagem de marcação acontece quando o programador insere um código PHP dentro de um script HTML. Quando a pagina é acessada, o PHP é executado em um servidor que gera o código HTML e o retorna como página carregada para o navegador \cite{Andrei2019b}.
PHP é amplamente usado no desenvolvimento web, desde sites e aplicações para a Internet, até mesmo extensões para WordPress e sistemas web, por exemplo. É uma das linguagens de programação mais versáteis e intuitivas existentes, sendo muito usada por programadores experientes e também por programadores iniciantes, por ter uma didática simples e de fácil compreensão.

\subsection{CSS}
CSS, acrônimo para Cascading Style Sheet, é uma linguagem criada em 1996 pela W3C (World Wide Web Consortium) e é responsável por adicionar estilos aos elementos das páginas. A necessidade de criação do CSS foi por um motivo bem simples. O HTML, linguagem de marcação também controlada e mantida pela W3C, não foi projetado para conter tags que ajudassem na formatação do estilo da página, por isso veio a necessidade de criação de uma ferramenta específica para isso. É responsável por mudar a cor de elementos, fontes, espaçamentos, ajustar a posição de elementos na página, ajustar tamanho de imagens entre outras funções que adicionam estilo às páginas \cite{Ariane2019}. 
A relação entre HTML e CSS é bem forte. Como o HTML é uma linguagem de marcação (o alicerce de um site) e o CSS é focado no estilo (toda a estética de um site), eles andam juntos. CSS não é tecnicamente uma necessidade, mas provavelmente você não gostaria de olhar para um site que usa apenas HTML, pois isso pareceria completamente abandonado.
\subsection{Weka}
Weka é um software que possui uma coleção de algoritmos de machine learning para tarefas de mineração de dados. Ele contém ferramentas para preparação, classificação, regressão, clustering, mineração de regras de associação e visualização de dados \cite{Weka}.
Weka é open source, desenvolvido em java e consolidado como o programa para mineração de dados mais usados por estudantes e professores em universidades. A ferramenta também é muito utilizada por aqueles que desejam aprender mais sobre mineração de dados. Através do Weka Explorer, interface gráfica do Weka, é possível realizar processos de mineração de dados de forma simples, e realizar a avaliação dos dados obtidos e a comparação de diferentes algoritmos. 
Weka utiliza preferencialmente bases de dados no formato texto. Por esta a maior parte das bases de dados usados para minerar dados através do Weka, são do formato ARFF ou CSV \cite{Goncalves2012}.