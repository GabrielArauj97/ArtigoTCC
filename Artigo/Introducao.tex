\newpage
\section{INTRODUÇÃO}
\pagenumbering{arabic}
O futebol é uma das modalidades esportivas mais famosas e disputadas do mundo. Trata-se de um esporte cujo objetivo transpor uma bola entre as balizas, que são as extremidades do campo, utilizando basicamente toques com os pés. Vence a partida a equipe que atingir o objetivo – que são chamados de gols - mais vezes na partida \cite{Sfeir2011}. Uma das principais razões pelo futebol se tornar uma febre mundial é o fácil entendimento das regras, o baixo custo e o fato de ser uma das modalidades mais empolgantes no meio esportivo. A última copa do mundo que foi jogada na Rússia no ano de 2018 atingiu uma audiência televisiva de mais de 3,5 bilhões de pessoas, batendo recorde de audiência, e somente a final entre França e Croácia atraiu 1,1 bilhões de telespectadores \cite{Chade2018}.

O alto nível de interesse das pessoas por esse esporte gera não apenas telespectadores, mas também muita movimentação financeira em torno dessa modalidade. A copa do mundo de 2018 teve lucro para a FIFA de 5,35 bilhões \cite{FIFA2018}. A movimentação financeira envolve patrocínios a clubes e seleções nacionais, venda de ingressos, produtos licenciados e transmissões por veículos de comunicação. Além é claro de transações de transferências envolvendo jogadores. Esse tipo de movimentação financeira atrai vários investidores que visão lucrar com o esporte. Sistemas computacionais que trabalham com a previsão de resultados e que auxiliam a minimizar os riscos e maximizar os lucros tornam-se então uma importante ferramenta de trabalho para o dia a dia do futebol \cite{Perin2013}.

O processo de scout é amplamente utilizado esportes, principalmente no futebol, para o registro, observação e análise do desempenho técnico e tático de equipes em partidas. Em esportes, o scout pode ser definido como uma técnica que consiste em analisar a partida, os momentos, os lances de um jogo para verificar o rendimentos das equipes. O scout é obejto de estudo em várias modalidades esportivas: basquete, vôlei, handebol, futebol americano, beisebol e futebol \cite{Martins2017}. A coleta de dados é utilizada não apenas para conhecimento do rendimento da própria equipe, mas também para o estudo de táticas e técnicas de equipes adversárias. Esses dados possibilitam mensurar quais são as principais características de uma equipe, identificando quais são suas principais jogadas, seus principais jogadores, sua organização tática e técnica, porém não existe uma padronização de quais dados devem ser coletados, o que deixa o processo muito abrangente \cite{Duarte2015}. O objetivo proposto é criar um software que seja capaz de acessar base de dados e compilar e projetar esses dados de maneira que auxilie o usuário a tomar decisões.

\subsection{OBJETIVO GERAL}
O objetivo deste trabalho é construir um software que consiga acessar diferentes bases de dados, compilar e projetar esses dados de maneira que auxilie o usuário no processo de tomada de decisão, assim facilitando e deixando o processo mais fácil de ser visualizado pelo usuário.

\subsection{OBJETIVOS ESPECÍFICOS}
\begin{itemize}
	\item  comparar as bases de dados existentes e compilar essas informações de maneira a projetar uma base de dados única;
	\item fornecer acesso ao usuários de informações referentes aos quatro elementos do processo de scout: espaço, tempo, jogador e fundamento;
	\item permitir ao usuário projetar e visualizar os relatórios com a compilação das informações proveniente da base de dados.
\end{itemize}