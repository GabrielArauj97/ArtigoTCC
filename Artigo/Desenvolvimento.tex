\section{DESENVOLVIMENTO}
\textit{Machine Learning}, termo em inglês para aprendizagem de máquina, é a capacidade de computadores aprenderem a tomarem decisões sem que sejam previamente programados para isso. Esse aprendizado, assim como acontece boa parte do aprendizado humano, se dá através de erros e acertos, que são obtidos através de algoritmos matemáticos. Esses algoritmos matemáticos trabalham classificando variáveis, por exemplo, queremos que um computador seja capaz de distinguir um ser humano de um gavião. Para que o computador seja capaz de fazer essa distinção, fornecemos variáveis que serão utilizadas pelos algoritmos matemáticos para realizar as classificações. Nesse caso, fornecemos variáveis como, possui asas, capacidade de falar, capacidade de voar, capacidade de caminhar, possui bico, põe ovos, entre outras. Assim, após o algoritmo analisar os dados, o computador poderá distinguir com facilidade humanos de gaviões. 
É importante frisar que a qualidade da classificação se dá pela qualidade das variáveis, porque se utilizarmos variáveis como, possui olhos, respira, possui dedos ou ate mesmo se alimenta, nesse exemplo isso seria inútil para o algoritmo, uma vez que ambos os elementos comparados possuem essas características, assim, além de ser desprezível para o algoritmo, poderia até ser prejudicial para sua execução. 
O objetivo desse projeto é criar um sistema que ajude no tratamento de dados para predição de partidas de futebol, sendo assim, utilizamos durante a execução do projeto bases de dados disponíveis no site Football-Data \cite{FootballData2019}, e algumas das variáveis que utilizadas foram:
% Please add the following required packages to your document preamble:
% \usepackage{booktabs}
\begin{table}[]
	\begin{tabular}{@{}|c|c|@{}}
		\toprule
		\multicolumn{2}{|c|}{\textbf{Variaveis utilizadas}}                              \\ \midrule
		FTHG & Full Time Home Goals (Gols do time da casa no jogo)                       \\ \midrule
		FTAG & Full Time Away Team Goals (Gols do time visitante no jogo)                \\ \midrule
		FTR  & Full Time Result (Resultado do jogo)                                      \\ \midrule
		HTHG & Half Time Home Team Goals (Gols do time da casa no primeiro tempo)        \\ \midrule
		HTAG & Half Time Away Team Goals (Gols do time visitante no primeiro tempo)      \\ \midrule
		HTR  & Half Time Result (Resultado do primeiro tempo)                            \\ \midrule
		HS   & Home Team Shots (Finalizações do time da casa)                            \\ \midrule
		AS   & Away Team Shots (Finalizações do time visitante)                          \\ \midrule
		HST  & Home Team Shots on Target (Finalizações certas do time da casa)           \\ \midrule
		AST  & Away Team Shots on Target (Finalizações certas do time visitante)         \\ \midrule
		HHW  & Home Team Hit Woodwork (Chutes na trave do time da casa)                  \\ \midrule
		AHW  & Away Team Hit Woodwork (Chutes na trave do time visitante)                \\ \midrule
		HC   & Home Team Corners (Escanteios do time da casa)                            \\ \midrule
		AC   & Away Team Corners (escanteios do time visitante)                          \\ \midrule
		HF   & Home Team Fouls Committed (Faltas cometidas pelo time da casa)            \\ \midrule
		AF   & Away Team Fouls Committed (Faltas cometidas pelo time visitante)          \\ \midrule
		HFKC & Home Team Free Kicks Conceded (Tiros de meta cedidos pelo time da casa)   \\ \midrule
		AFKC & Away Team Free Kicks Conceded (Tiros de meta cedidos pelo time visitante) \\ \midrule
		HO   & Home Team Offsides (Impedimentos do time da casa)                         \\ \midrule
		AO   & Away Team Offsides (Impedimentos do time visitante)                       \\ \midrule
		HY   & Home Team Yellow Cards (Cartões amarelos do time da casa)                 \\ \midrule
		AY   & Away Team Yellow Cards (Cartões amarelos do time visitante)               \\ \midrule
		HR   & Home Team Red Cards (Cartões vermelhos do time da casa)                   \\ \midrule
		AR   & Away Team Red Cards (Cartões vermelhos do time visitante)                 \\ \bottomrule
	\end{tabular}
\end{table}

\newpage
Essas variáveis apresentam importância significativa para os algoritmos, uma vez que podem determinar o andamento da partida e assim o resultado final.
A imagem a seguir apresenta os resultados da classificação com as seguintes variáveis: FTHG, FTAG, FTR, HTHG, HTAG, HTR, HS, AS, HST, AST, HF, AF, HC, AC, HY, AY, HR, AR.

\begin{figure}[htbp]
	\begin{center}
		\includegraphics[width=.9\linewidth]{imagens/dados_completos.png}\\
	\end{center}
	\caption[Classificação com variáveis essenciais]{Classificação com variáveis essenciais}
	\label{fig:logo}
	%\legend{Fonte: Próprio Autor}
\end{figure}

E essa imagem apresenta os resultados com as seguintes variáveis: FTR, HTHG, HTAG, HS, AS, HST, AST, HF, AF, HC, AC, HY, AY, HR, AR.

\newpage

\begin{figure}[htbp]
	\begin{center}
		\includegraphics[width=.9\linewidth]{imagens/dados_incompletos.png}\\
	\end{center}
	\caption[Classificação com ausência de algumas variáveis essenciais]{Classificação com ausência de algumas variáveis essenciais}
	\label{fig:logo}
	%\legend{Fonte: Próprio Autor}
\end{figure}


Como é possível notar, a ausência das variáveis FTHG, FTAG, HTR fez com que o resultado final da classificação de instancias corretas caísse mais de 44 pontos percentuais, pois tais informações (gols do time da casa, gols do time visitante e resultado do primeiro tempo) são essenciais para definir o resultado final da partida.

Para auxiliar esse processo de escolha de variáveis, foi criado um sistema que funciona da seguinte maneira.

A página inicial é composta por dois elementos, sendo um botão do tipo input, que permite ao usuário a escolha da base de dados presente na máquina, e um outro botão do tipo submit que confirma o envio da base de dados para a pasta do projeto, para que seja utilizada nas demais telas.

\begin{figure}[htbp]
	\begin{center}
		\includegraphics[width=.9\linewidth]{imagens/home.png}\\
	\end{center}
	\caption[Página inicial do sistema]{Página inicial do sistema}
	\label{fig:logo}
	%\legend{Fonte: Próprio Autor}
\end{figure}
\newpage
A segunda página é composta por uma tabela que lista todos os dados da base de dados, permitindo a visualização completa desses dados. Nessa tela também estão presentes um elemento select multiple, permitindo que o usuário escolha as variáveis que irá usar para a construção de uma nova base de dados apenas com os dados selecionados pelo usuário.
\begin{figure}[htbp]
	\begin{center}
		\includegraphics[width=.9\linewidth]{imagens/listagem.png}\\
	\end{center}
	\caption[Listagem dos dados do arquivo selecionado]{Listagem dos dados do arquivo selecionado}
	\label{fig:logo}
	%\legend{Fonte: Próprio Autor}
\end{figure}
\begin{figure}[htbp]
	\begin{center}
		\includegraphics[width=.9\linewidth]{imagens/listagem2.png}\\
	\end{center}
	\caption[Tabela com os dados]{Tabela com os dados}
	\label{fig:logo}
	%\legend{Fonte: Próprio Autor}
\end{figure}

\newpage

A terceira página é responsável por listar os dado que foram selecionados na página anterior e gerar a base de dados limpa. Essa base de dados fica salva em uma pasta chamada Arquivos, localizada dentro da pasta do projeto.
\begin{figure}[htbp]
	\begin{center}
		\includegraphics[width=.9\linewidth]{imagens/dados.png}\\
	\end{center}
	\caption[Listagem dos dados do arquivo selecionado]{Listagem dos dados do arquivo selecionado}
	\label{fig:logo}
	%\legend{Fonte: Próprio Autor}
\end{figure}
