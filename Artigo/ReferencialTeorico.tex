%\thispagestyle{empty}
%Modifique a estrutura dos capítulos e seções de acordo com a necessidade do seu trabalho
\section{REFERENCIAL TEÓRICO}
Em uma partida de futebol o número de anotações que se pode fazer para que seja possível sua correta descrição é bastante elevado \cite{Constantinou2013}. Se for levado em consideração outros fatores extracampo temos novamente um alto número de componentes envolvidos. Dessa maneira a definição de quantas e quais características serão utilizadas para se realizar a predição dos resultados de partidas se torna bastante complexa \cite{Tax2015}. Na literatura é possível encontrar trabalhos que levam em consideração uma série de fatores tais como dados de fundamentos obtidos por \textit{scout} como em \citeonline{Sfeir2011}. Em \citeonline{Parinaz2013} utilizou fatores de logísticas tais como distância percorrida entre duas partidas subsequentes \cite{Tax2015} ou fatores psicológicos como em \citeonline{Duarte2015}.

O processo \textit{scout} apesar de bastante difundido e utilizado não é padronizado no que diz respeito ao número de fundamentos que serão coletados \cite{Pendharkar2000}. Em Brooks \cite{Brooks2016} é utilizado um conjunto de 7 fundamentos obtidos por scout e um conjunto de 6 características de logísticas para se obter uma correta predição. Já em \citeonline{Ulmer2013} são utilizados 9 fundamentos obtidos por \textit{scout}. Em \citeonline{Igiri2015} são utilizadas como características 23 fundamentos coletados por \textit{scout}. Em \citeonline{Hucaljuk2011} são utilizados 10 fundamentos obtidos por \textit{scout}. \citeonline{Tax2015} utiliza 19 características obtidas por \textit{scout}, 12 de logística e 12 baseadas em sites de apostas. Enquanto que \citeonline{Duarte2015} utiliza um conjunto de 12 características obtidas por \textit{scout} e 12 características envolvendo aspectos psicológicos das equipes. 
 
É importante ressaltar mais uma vez que não existe uma padronização para as características envolvidas com os trabalhos existentes na literatura. Pode-se afirmar, entretanto que as características estão ligadas ao tipo de informação que se necessita coletar sobre um ou mais aspectos que envolvem uma partida de futebol \cite{Tax2015}. Não é possível ainda afirmar que uma outra metodologia é superior a outra, mas que são complementares. Para o caso desse trabalho, o mesmo deverá ser capaz de: comparar as bases de dados existentes e compilar essas informações de maneira a projetar uma base de dados única.